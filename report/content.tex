\section{Introduction}
\subsection{Motivation}
In the space of \ac{HPC}, a significant trend is from more compute intensive tasks to more data intensive tasks is taking place. This change necessitates better data-management tooling such as data lakes or data warehoueses. Given that \ac{HPC} computes on raw data, data lakes are a more natural fit. Efficient metadata management requires a great metadata store as well as support for full text searches.

In this report, the viability of Elasticsearch\footnote{\url{https://www.elastic.co/elasticsearch}} for \ac{HPC} load is being evaluated. Elasticsearch benefits from many years of tooling development, making it a more pragmatic choice than building an own application around full text search engines like Apache Lucene\footnote{\url{https://lucene.apache.org/}} or Tantivy\footnote{\url{https://github.com/quickwit-oss/tantivy}}.

While elastic provides rally\footnote{\url{https://github.com/elastic/rally}}, an in-house benchmark suite used for performance regression testing \cite{es_benchmarking}, after rigorous internal testing it was found out that thespian\footnote{\url{https://thespianpy.com/doc/}}, its internal actor framework, did not scale up to more than 60 nodes, which was not sufficient for previous large scale stress testing. Thus, for this report a new benchmarking framework was designed, using reliable \ac{HPC} native technologies such as \ac{MPI}.

In addition, the data lake related use cases at the GWDG include spawning Elasticsearch instances on-demand. For this dynamic deployment, a containerized Elasticsearch cluster based on the underlying SLURM\footnote{\url{https://slurm.schedmd.com/documentation.html}} and \ac{MPI} is needed. It should auto-configure itself, in order to be IP-agnostic.
\subsection{Goals and Contributions}
The goals of this report are twofold: First, providing and implementing an stateful architecture to dynamically spawn and respawn any sized Elasticsearch cluster without any previous manual configuration. Second, design a highly-scalable, \ac{HPC}-native, distributed Elasticsearch benchmarking framework for both ingestion and querying performance. 

In order to achive these goals, the following contributions were made:

\begin{itemize}
\item Design and implementation of a zero-configuration workflow to spawn a rootless, containerized Elasticsearch cluster of arbitrary size within an SLURM-allocated \ac{MPI} environment.
\item Design and implementation of a highly scalable, distributed benchmarker for ingestion performance
\item Design and implementation of a highly scalable, distributed benchmarker for query performance with custom benchmark scenarios using an \ac{JSON} based \ac{DSL}.
\item Benchmark an dynamically spawned Elasticsearch cluster on \ac{HPC} using a canonical dataset common in existing literature.
\end{itemize}
\subsection{Structure}
This report is structured as follows: In section 2, Elasticsearch, the full text search engine and NoSQL database used, is introduced. After that, section 3 covers the related work starting with classical load generation before focussing on the literature around Elasticsearch benchmarking. In section 4, the methodology and design of all components is covered. First, it covers the design and internal workflow of the aforementioned cluster spawning mechanism. Then, it covers both the ingestion and query benchmarkers. Lastly, it focuses on the benchmark design by showing the steps required to perform the benchmarks and the reasoning behind the corpus and query design used in this specific benchmark. Lastly, in section 5 the results of the benchmark is shown, before going into further discussion and analysis in chapter 6. Concluding in chapter 7, the results are summarized and possible future work is shownis shown.

\section{Background}
\subsection{Elasticsearch}
Elasticsearch is a distributed search engine initially developed in 2010. Since it stores its data in a document model, it can also be seen as a NoSQL database. The full text search is internally relies on the Apache Lucene library. It is mainly used for its full text fuzzy search capabilities and its \ac{JSON}-based REST interface. It is used by many large websites such as Wikipedia\footnote{Wikipedia also used Lucene beforehand.}, Netflix, Stackoverflow, and LinkedIn.

In Elasticsearch, a collection of \emph{documents} are stored in an \emph{index}. Documents are equivalent to, and can be sent in the form of, \ac{JSON} objects. They can be nested. Each index has a \emph{schema}, which is a type mapping for each of the key-value pairs contained in the index\footnote{Akin to a SQL database definition.}. This mapping can either be static or dynamically guessed based. The index data can be \emph{queried} using Elasticsearchs own \ac{DSL}, again relying on \ac{JSON} and REST as the foundational technology.

In 2021, due to a license change from Apache 2.0 to a more permissive license, OpenSearch was created as an Elasticsearch fork, which is maintained by several companies such as AWS.

\section{Related Work}
While the topic of Elasticsearch benchmarking is more sparsely covered, a lot of previous work around general HTTP API benchmarking exists. The most used load generator is Apache JMeter \cite{jmeter}, a sophisticated graphical load tester that supports many protocols such as HTTP(S), SOAP, or LDAP. Only relying on \ac{CLI} interface, wrk \cite{wrk} provides a more simple and trivial popular alternative. For a more scriptable alternative, the Javascript based Grafana k6 \cite{k6} gained a lot of popularity over recent years.\\

For benchmarking Elasticsearch, the main tool is the aforementioned rally \cite{rally}, a microbenchmarking framework developed by elastic. While it can be run on a single node, it also supports distribued benchmarking through the Thespian actor system.

Different benchmarking scenarios are defined as so-called \emph{tracks}. Every track contains one or more \emph{corpora}, containing \ac{NDJSON} objects as documents. All tracks are available on Github \cite{rallytracks}. Each track contains many \emph{operations} such as ingestion or specific queries, which are then structured into a \emph{challgenge's} \emph{schedule} in a fork-join model. This means that Rally can be extended without editing the source code.

The rally framework is actively used in literature for benchmarking Elasticsearch clusters \cite{rallyusecase1} \cite{rallyusecase2} \cite{rallyusecase3}. As mentioned in the introduction, it was not a viable choice for this paper due to the underlying actor framework not scaling into more 128 nodes in previous experiments.\\

In the field of \ac{TSDB} benchmarking, the canonical solution to compare the performance of different \acp{TSDB} is Timescale's \ac{TSBS} \cite{tsbs}, a fork of Influx influxdb-comparisons \cite{ifdbcomp}. The benchmark started as a comparison between InfluxDB\footnote{\url{https://www.influxdata.com/}} and Elasticsearch \cite{ifdbes}. Unfortunately, both repositories seem to be mostly abandoned.\\

Furthermore, most of the benchmark comparisons between NoSQL databases are done by database vendors themselves, resulting in conflicting financial interests.

\section{Methodology and Design}
The Methodlogy and Design section is split into four parts: First, the autoconfigured Elasticsearch cluster spawner is presented. After that, the next two subsections cover the distributed ingestion and query benchmarker respectively, going into a lot of reasoning and technical detail. Lastly, the overall benchmark design used for this report is discussed, focusing mainly on the high level benchmark workflow as well as the corpus and query design.

\subsection{Dynamic Cluster Creation Based on MPI Communicator}
This section presents an automated approach to to configuring and spawning a multi node Elasticsearch cluster based solely on the \ac{MPI} environment set up by SLURM. It dynamically fetches the different hosts, i.e. it is not required to know the hostnames or IPs beforehand. The cluster is very portable since it is using Singularity \cite{singularity} containers as a elasticsearch host. Any cluster size larger or equal to two nodes is supported.

Furthermore, it supports \emph{statefulness}, which means that the same cluster can be re-spawned on other nodes\footnote{With the same cluster size.} without requiring a re-ingest, i.e. keeping the complete cluster state and configuration. Due to the aforementioned containerization, this also works while changing both the hardware and IP addresses. The only limitation is that it only supports one elasticsearch node per host OS. This is by design, as we use hostname-based resolution instead of IP-based resolution to be agnostic to the \ac{NIC} used\footnote{Being \ac{NIC} agnostic implies that it supports both ethernet and infiniband.}. 

This automatic containerization has the big advantage that it can be embedded into any kind of job pipeline. While most web services require a continuously running search engine, it is common in \ac{HPC} that applications are spawned on demand only when needed and teared down once the computation is performed. More importantly, it allows for Elasticsearch to be implicitly spawned as a dependency for other, more cloud-native applications running in \ac{HPC}.\\

The high level idea is as follows: For discovery and information communication, \ac{MPI} is used. Furthermore, based on the world size, the number of master eligable nodes is decided\footnote{If $N \leq 3$ then 1 master eligable node, otherwise 3 master eligable nodes. Note that only one master is active at a time. A odd number was chosen to prevent split brain.}. After that, each node creates a config and runtime environment for itself. Lastly, each process spawns its container using its newly generated config.\\

For better portability and reproducability the cluster generator uses Singularity containers internally\footnote{Note that, due to Elasticsearch JDK problems, Singularity has to be started with \texttt{-{}-cleanenv}. Therefore environment variables get ignored within the container.}. The container is based on the official Ubuntu 22.04 docker image\footnote{\url{https://hub.docker.com/_/ubuntu}}, only adding packages for debugging and maintenance. Elasticsearch itself is not part of the image and completely bind-mounted in; there are multiple reasons for this: First, unlike docker, Singularity containers are always immutable, so the program state itself has to always be bind-mounted in. Furthermore, Elasticsearch expects to be the owner of its folders. If Elasticsearch is not the folder owner, it disables multiple features such as internal auto-configuration. Since Singularity is rootless, the correct uid for the folders can't be enforced. Therefore, the most straight forward and stable solution is bind-mounting it in.\\

On a high level, the spawner works as follows:
\begin{itemize}
  \item First, check if the elasticsearch index data from last run can be reused. This is the case if the number of nodes stayed the same. If not, create a new elasticsearch into a temporary directory.
  \item \texttt{MPI\_GATHER} a list of tuples \texttt{(rank, hostname)} into the root rank 0.
  \item For each node, the root creates/updates the config; the other nodes are waiting. Note that through updating the config instead of creating a new one the cluster stays intact, which is how the statefulness is implemented. An example config can be found in the appendix.
  \item Lastly, each rank starts the immutable singularity container with the new config and previously created elasticsearch mounted in.
\end{itemize}

\newpage

\subsection{Distributed Ingestion Benchmarker}
This section introduces the distributed, \ac{MPI}-based ingestion benchmarker. It is used both for providing a fast way to ingest an \ac{JSON}-formatted corpus into an elasticsearch cluster as well as measuring the performance of write operations in throughput as well as latency. The benchmarker itself is very I/O optimized, using so-called offset caching for reducing redundant filesystem load between workers. It supports statically typed index definitions, configurable bulk size as well as a configurable number of shards. It supports all corpora designed for elastics own rally benchmarker by using \ac{NDJSON} as an input format\footnote{Another big advantage of \ac{NDJSON} is that downscaling the benchmark is trivial using\\\texttt{head -n <NEW\_NUMBER> original.json > downscaled.json}}. The \ac{CLI} interface with all its features can be found in the appendix.\\

The benchmark can be split into three phases:

\paragraph{Setup:}
Note that those steps are done by only the root / rank 0.
\begin{itemize}
  \item Create the offset cache.

    The offset cache is needed for the following reason: In order to do the ingestion in a distributed manner, the bulk ingest load has to be split evenly between all nodes. This is done by giving each rank (approximately) the same number of documents in the corpus file. Note that \ac{NDJSON} implies one document per line. For $N$ nodes and $L$ lines, each $i$ gets the range
    \[
      \left[ \frac{i}{N} \cdot L, \frac{i+1}{N} \cdot L \right)
    \]
    But this requires that the number of lines have to be known, which implies reading the whole file at least once. After the range has been computed, the file has to be read a second time in order to find the starting byte to seek to. This is needed since the \ac{JSON} documents have a variadic size in bytes; one can't just compute the $i$-th document through $doc\_size \cdot i$.

    The corpora are often huge; the \texttt{nyc\_taxis} corpus used in this report is around 75GB in size with over 160,000,000 documents. This would create a lot of I/O load if every node would do this every time the benchmark runs! 

    So instead, the root computes all offsets once and saves it into a \texttt{.offsets.json} file, which can be reused in future benchmarks, removing all redundant work. This optimization is possible whenever the number of load generators stay the same.

    On a technical level, this is done as follows:
    \begin{enumerate}
      \item Iteration 1: Count the number of lines.
      \item Compute the starting and ending line for each rank using the total number of lines.
      \item Iteration 2: Find the byte offsets for each rank.
      \item Save everything into a \texttt{.offsets.json} file.
      \item Validate that, starting at the current byte, the line of each rank has a complete \ac{JSON} document.
    \end{enumerate}
    An example \texttt{.offsets.json} file can be found in the appendix.

  \item Create an (empty) Elasticsearch index. It is created using the following settings:
    \begin{itemize}
      \item Strict type mappings for reproducability. Elasticsearch allows for dynamic schemas, which means that the datatype is interpreted when the first data arrives. When using a distributed benchmarker, it is not clear which rank sends the first bulk ingest request. Thus, indeterministic or unexpected behaviour could occur. So instead, using elasticsearchs own type sytem\footnote{\url{https://www.elastic.co/guide/en/elasticsearch/reference/current/mapping-types.html}} and \ac{DSL}\footnote{\url{https://www.elastic.co/guide/en/elasticsearch/reference/current/mapping.html}} the type mapping can be defined statically and used by the benchmarker.
      \item The number of shards explicitly set, defaulting to one shard per elasticsearch node. This means that every node gets data while still keeping the sharding complexity as trivial as possible. This is configurable via the \ac{CLI} interface.
      \item Explicity disable caching through \texttt{requests.cache.enable}\footnote{\url{https://www.elastic.co/guide/en/elasticsearch/reference/current/shard-request-cache.html}}
    \end{itemize}
\end{itemize}
At the end of the setup phase, the offsets are sent to all workers using \texttt{MPI\_BROADCAST}.

\paragraph{Benchmark:} This work is done by each rank including the root.
\begin{itemize}
  \item Each rank computes to which elasticsearch node it should send the data. 

    It is assumed that the user understands that this distribution is a problem that has to be solved. Thus, it is assumed that the \ac{MPI} world size is a multiple of the number of elasticsearch cluster nodes\footnote{It also works when this isn't the case, although the distribution is less optimal.}. With that in mind, the distribution is calculated by \texttt{rank \% N} with \texttt{\%} being the modulo operation and \texttt{N} the number of elasticsearch cluster nodes

  \item First, seek to the starting byte based on the rank.
  \item Create and send the requests blockingly, as fast as possible. Track each response time.

    The requests are sent in bulk using elasticsearches Bulk API\footnote{\url{https://www.elastic.co/guide/en/elasticsearch/reference/current/docs-bulk.html}} with a default bulk size of 1000 documents, configurable via CLI parameter.

    The measurements are done with utmost care to be as precise as possible and minimize the overhead created by the benchmarker itself. Thus, the delta recorded only measures the time directly before and after the HTTP request, removing the query building overhead.
  \item Once all data was sent, the workers wait at an \ac{MPI} barrier for the other workers to finish.
  \item At the end, all data is gathered at the root process.
\end{itemize}

\paragraph{Teardown:}
Once the data is gathered, the root dumps it into a \ac{JSON} file.

\subsection{Distributed Query Benchmarker}
While the last section focused on the ingestion (write) performance, this section will focus on benchmarking the query (read) performance of an elasticsearch cluster. The benchmarker also generates its load in a distributed manner using the \ac{MPI} environment provided by SLURM. It measures the docuemnts per second as well as the request latency of an cluster when put under load. The benchmark is structed using a fork-join like model. 

The distributed query benchmarker has the following features:
\begin{itemize}
  \item Fully configurable by a \ac{DSL}-like \ac{JSON} standard; no hard coded scenarios.

    The \ac{DSL} uses Elasticsearch embeds the elasticsearch query language internally, making it very accessible to all elasticsearch users. Furthermore, this allows extending the benchmarking scenarios without the need of editing the source code. Since the language is a simplification of rallys \ac{DSL}, elastics official benchmarks can be ported easily. An example of a ported rally benchmark can be found in the accompanying repository.

    Also, an example input file using the custom \ac{DSL} can be found in the appendix.
  \item Measuring raw performance by bypassing the cache\footnote{\url{https://www.elastic.co/guide/en/elasticsearch/reference/current/shard-request-cache.html\#_enabling_and_disabling_caching_per_request}}.
  \item Supporting a test mode for verifying the correctness of custom benchmarks
  \item Parsing the elasticsearch response instead of just relying on the HTTP response like normal HTTP benchmarker. This is done to count the number of returned documents.
  \item Supporting multiple, alternating queries in a single fork-join task for creating a more realistic load.
\end{itemize}

On a technical level, each benchmark starts with the following prepararation.
\begin{itemize}
  \item First, all \ac{CLI} arguments and the benchmark description are parsed. The benchmark description is written in the aforementioned, JSON-based custom DSL with the Elasticsearch Search API Syntax embedded into it. 
    \begin{itemize}
      \item Note that an example query benchmark description as well as the \ac{CLI} interface structure of the both benchmarkers can be found in the appendix.
    \end{itemize}
  \item After that, every rank calculates which elasticsearch server it sends its load to. See the ingest benchmarker section for a more detailed description on how this selection algorithm works.
\end{itemize}

Once all ranks are ready, the actual benchmark starts. For each of the fully disjunct benchmark steps, the following steps are performed:
\begin{itemize}
  \item First, the root node, i.e. \ac{MPI} rank 0, waits for the cluster health to be green while the other nodes wait at a \ac{MPI} barrier, ready to benchmark. This is done in case the data was just ingested or the cluster was just started on-demand.
  \item As mentioned above, each benchmark step can contain multiple queries, which will then be executed in an alternating manner to simulate a more varied load on the server. Since all ranks start executing the queries at the same time, it could happen that all ranks are in-sync, sending the exact same query at the same time, which would not be a realistic load. To prevent this, each rank creates a random permutation of all queries for this step.
  \item If the warmup time is set:
    \begin{itemize}
      \item Send the next query, throw away the result. This is needed to get the caches filled before starting the measurements; therefore reducing the variance of the result data. The index is configured to not do any caching on an Elasticsearch level, however the OS will still do caching, both on a CPU L1/L2/L3 level as well as I/O caching through the page and buffer cache.
      \item Sleep between results if a waiting time is configured.
    \end{itemize}
  \item Once the warmup is done, the following steps are done until the configured execution time is reached:
    \begin{itemize}
      \item Select the next query
      \item Track the current time right before sending the request
      \item Do the request containing the query against the search API endpoint. Note that the cache is explicitly disabled, both on an index level and on an request level\footnote{\url{https://www.elastic.co/guide/en/elasticsearch/reference/current/shard-request-cache.html\#_enabling_and_disabling_caching_per_request}}.
      \item Track the time right after the response was received, before processing the result. This is done for two reasons: First, it minimizes the measurement overhead created by the benchmarker itself and isolates the time Elasticsearch needed for creating and sending the response. Second, it increases the likelyhood of our operation to be atomic, i.e. that the OS scheduler will preempt the process by giving the CPU time to another process before the measurement was finished.
      \item Next, it will process and record the result based on the HTTP response code. If it was successful\footnote{i.e. a 2xx HTTP response code}, it will count the number of received documents and will save a tuple \texttt{(latency, docs count)} for the current query and rank. If the request was unsuccessful, it saves the HTTP code for the current query and rank. Note that this all happens in Memory, no slow I/O is done in order to minimize the benchmarker overhead.
      \item Lastly, sleep between results if a waiting time is configured.
    \end{itemize}
\end{itemize}

Once all measurements are created, the results of all ranks get merged at the root using \ac{MPI} gather as well as some data transformation. The resulting output format can be seen in the appendix. To finish the execution, the root dumps the JSON-formatted results at a path specified via a \ac{CLI} paramter.

\subsubsection{Test Mode}
Designing a database benchmark is complicated; it should be very clear both what the expected result is and whether the query actually produces that result. In order to verify the correctness of a given benchmark, a test mode was developed. This test mode can be run with \texttt{-{}-test-mode}.

The purpose of the test mode is to give a short overview to check whether the raw Elasticsearch responses contain the results expected. Therefore it is sufficient that the test mode only runs on the root rank. The test mode itself is pretty trivial: It goes through each benchmark step, runs each query once, verifying that the response code is successful, and then prints the request input as well as the response output as prettified JSON into the terminal for further manual inspection.

While designing several benchmarks, it helped finding broken queries generating empty responses.

\subsection{Benchmark Design}
\subsubsection{High Level Benchmark Workflow}
To create and run a benchmark, the following steps have to be done:
\begin{enumerate}
  \item Choose a dataset or create a synthetic one. The dataset has to be formatted in the \texttt{ndjson} format, which is defined as one JSON object per line. Datasets for many common Elasticsearch use cases can be found in Elastics rally-tracks repository \cite{rallytracks}.
  \item Define the data type mappings for each corpus attribute using the Elasticsearch index mapping syntax\footnote{\url{https://www.elastic.co/guide/en/elasticsearch/reference/current/properties.html}}. An example can be found in the repository.
  \item Design the query document. This basically consists of embedding the elasticsearch search API queries into a bigger JSON structure defining the benchmark steps.  Note that the queries itself do not have to be altered, thus they can also easily be constructed using cURL. Since Elastics rally uses also uses the search API syntax, their benchmarks can easily be ported. For each benchmark step the warmup time, execution time, and sleep time between each request can be defined optionally. The query document format can be found in the appendix, a ported benchmark from rally can be found in the repository.
  \item Spawn up the automatically configured elasticsearch cluster using the \ac{MPI}-based cluster creator in a given SLURM-environment using \texttt{mpirun}.
  \item Run the distributed ingest benchmarker with the previously created corpus and type mapping using \texttt{mpirun}.
  \item Run the distributed query benchmarker with the previously created query document using \texttt{mpirun}.
  \item Analyze all results. A Jupyter notebook in the repository can be used as a starting point.
\end{enumerate}

\subsubsection{Corpus and Query Design}
The benchmark created for this report is a port of rallys \texttt{nyc\_taxis} track \cite{nyctaxis}. Its corpus contains New York taxi data, more specifically all rides that have been performed in yellow taxis in New York in 2015. This data gets published every year by the NYC Taxi and Limousine Commision and can be freely downloaded \cite{tlcdata}. An example corpus document can be found in the appendix.\\

This benchmark was chosen as it is one of the two rally benchmarks most often used in literature, \texttt{geonames} and \texttt{nyc\_taxis}. In the literature, both of these benchmarks have a specific purpose. 

\paragraph{\texttt{nyc\_taxis}} is mainly used as a scaling test because of its big corpus size with over 165 million documents and comparatively big document size. The documents are very numeric, which makes the data set great for benchmarking range queries and aggregations such as histograms. It can't be used for thorough benchmarking of string matching capabilities.

\paragraph{\texttt{geonames}} is mostly used as a regression test for various features. It has a very balanced document, containing text, keywords, numbers as well as geolocations. The main advantage of \texttt{geonames} is the amount of queries designed for it, including text and keyword matching, several aggregation, scroll API, Elasticsearchs expression and painless scripting languages and many more niche features of Elasticsearch.

Since the internal use case at the GWDG is mainly around fuzzy matching and range queries running on large corpora in an \ac{HPC} environment, the \texttt{nyc\_taxis} track is a better fit.\\


For the measurements done in this report, a full query benchmark was designed, loosely based on the one provided by Elastic. Note that the main purpose of this benchmark is to provide a starting point on how to design Elasticsearch queries for this benchmarker. 

It contains the following benchmark steps:
\begin{itemize}
  \item Simple, non-filtering \texttt{match\_all}\footnote{\url{https://www.elastic.co/guide/en/elasticsearch/reference/current/query-dsl-match-all-query.html}} requests, with response size of 10, 100, 1000, or 10000 documents. This is done to benchmark the base I/O performance and response size scaling.
  \item Next a \texttt{range} query\footnote{\url{https://www.elastic.co/guide/en/elasticsearch/reference/current/query-dsl-range-query.html}}, again scaling up the number of documents between each step. The range query is the exact same one also used by Rally.
  \item After that, it alternates the \texttt{range} and \texttt{match\_all} query in the same benchmark step. This is done to simulate a more varied and realistic load.
  \item In order to measure the latency change given less load, the \texttt{match\_all} query gets executed with 0.02, 0.05, 0.1, 0.2, 0.5, and 1 second sleep between each request.
  \item Lastly, multiple histogram aggregations, either with automatic or manual bucketing, are being executed, all designed by Elastic. This allows us to measure aggregation performance while also being comparable to other benchmarks made with Rally.
\end{itemize}

\section{Test Run and Analysis}
% Note that the goal here is to confirm that it works properly, not to actually analyze the performance!
% - Setup
%   - Running on Emmy
%     - mention HW here, atm 3 gcn2 nodes
%     - we use ethernet interconnect, not OPA
%   - Ubuntu 22.04 container
%   - Elasticsearch 8.11.0 with its default java version (openjdk 21.0.1)
%   - Configured to not be cached
%   - Python 3.9.16
%   - OpenMPI 4.1.4 compiled with ICC
%
% The results should mainly just show that the latency and throughput scales with the number of load generators per node.
% maybe also show how the range query scales with output size for a given number of load generators
% thats enough, the raw data can be found in the repo

\section{Challenges}
The benchmarker still has two problems, both of which are not trivial to solve optimally.
% - Problem: result size
%   - Note that default size are the first 10 elements if not specified otherwise
%   - specified by `index.max_result_window`, usual 10000
%   - this means that we can only get the first 10k hits
%   - Theoretically one can offset with `from + size`, but `from+size <= index.max_window`
%   - Possible solutions:
%      - Just use the first 10k, ignore it
%        - Thats what elastic does with their official benchmarks, they only do size=1000 even with a 165 mil doc dataset
%     - use the scroll API. 
%       - Problem: It is stateful and requires a setup `search` query
%       - blows up benchmark complexity
%     - Use the `search_after` parameter for the serach API
%       - Same problem, also requires an ID to be selected before, very complex especially load distributed
% - Problem: our ES selection algorithm could result in a suboptimal network topology (i.e. long distance in fat tree)

\section{Conclusion}

\subsection{Future Work}
% - We could have also used the `took` key in the search API response: <https://www.elastic.co/guide/en/elasticsearch/reference/current/search-search.html#search-api-response-body>
%   - see what it includes and excludes
%   - describe since how little overhead we have measuring (first measure, then validate) how we trust ourselves more than making sure that elasticsearches overhead is predictable
% - Circumvent the 10k limitation using either scroll API or `search_after`
% - Bring it into production
% - find a way to make the load distributor to cluster node mapping network topology aware
% - doing larger scale benchmarks

% TODO FUTURE WORK DATA LAKE ENCRYPTION MENTION THAT WE ACTUALLY DO THAT RN
